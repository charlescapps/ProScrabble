\documentclass{article}

\usepackage{fullpage, graphicx, amssymb, amsmath, amsfonts, wrapfig, subfigure, listings, color, hyperref}
\usepackage[usenames,dvipsnames]{xcolor}


\usepackage[ocr-a]{ocr}
\usepackage[T1]{fontenc}



\newcommand{\nl}{\vspace{\baselineskip}}
\newcommand{\tab}{\hspace*{2em}}
\newcommand{\bo}[1]{\textbf{#1}}
\newcommand{\tm}[1]{\textrm{#1}}
\newcommand{\imp}{\longrightarrow}
\newcommand{\f}{\forall}
\newcommand{\e}{\exists}
\newcommand{\AND}{\wedge}
\newcommand{\OR}{\vee}

%Some settings to make lstlistings pretty
\lstset{ 
language=C,                % choose the language of the code
basicstyle=\footnotesize,       % the size of the fonts that are used for the code
numbers=left,                   % where to put the line-numbers
numberstyle=\footnotesize,      % the size of the fonts that are used for the line-numbers
stepnumber=1,                   % the step between two line-numbers. If it is 1 each line will be numbered
numbersep=5pt,                  % how far the line-numbers are from the code
backgroundcolor=\color{white},  % choose the background color. You must add \usepackage{color}
showspaces=false,               % show spaces adding particular underscores
showstringspaces=false,         % underline spaces within strings
showtabs=false,                 % show tabs within strings adding particular underscores
frame=single,   		% adds a frame around the code
tabsize=2,  		% sets default tabsize to 2 spaces
captionpos=b,   		% sets the caption-position to bottom
breaklines=true,    	% sets automatic line breaking
breakatwhitespace=false,    % sets if automatic breaks should only happen at whitespace
%escapeinside={\%}{)}          % if you want to add a comment within your code
}

\begin{document}
\ttfamily

\begin{center}
\Huge{ACM Challenge: Scrabble AI}\\

\end{center}
\large
Your mission, should you choose to accept it, is to create a Scrabble AI that can find the best Scrabble move for any Scrabble board and tiles in hand. Your program will read in a Scrabble board and tiles in hand from stdin, then output the best Scrabble move to stdout. 

(If you are awesome, you will then create an AI that takes into account rack balancing and other strategies.)

\section{Board Format}

Sample test files are available at \verb=somedir/testcases/test*.txt=

They look like this: 

\begin{verbatim}
SCRBL_RACK: XARTH**
SCRBL_BOARD:
_ _ _ _ _ _ _ _ _ _ _ _ _ _ _ 
_ _ _ _ _ _ _ O*_ _ _ _ _ _ _ 
_ _ _ _ _ _ _ U P _ _ _ _ _ _ 
_ _ _ _ _ _ _ T O _ _ _ _ _ _ 
_ _ _ _ R _ _ W E B _ _ _ _ _ 
_ _ _ _ E _ _ A T E _ _ _ _ _ 
G _ _ _ Q _ _ R _ E*_ _ _ _ _ 
U _ S Q U A*W S _ S _ _ _ _ _ 
L A W _ I*_ _ _ _ W _ _ _ _ _ 
_ H A _ T _ _ _ _ A _ _ _ _ _ 
_ _ R _ A _ _ _ _ X _ _ _ _ _ 
_ _ T*_ L*_ _ _ _ _ _ _ _ _ _ 
_ _ H _ _ _ _ _ _ _ _ _ _ _ _ 
T E S S E R A _ _ _ _ _ _ _ _ 
_ _ _ _ _ _ _ _ _ _ _ _ _ _ _ 
\end{verbatim} 

A `*' in the rack string indicates you have a blank tile. A `*' on the board following a letter indicates that letter was played as a blank tile (and so it counts for 0 points.) For example, in this board someone played BEESWAX going south and used a blank tile for an `E'. 

\section{Move Format}
After reading in this file, your AI will then output a scrabble move. It can output extra junk so long as a move in this format appears on some line. 

Move Format: ``\verb=SCRBL_MOVE: (tiles_used) (ROW, COL, word_played, EAST | SOUTH)="

Example: ``\verb=SCRBL_MOVE: (EQURRY) (4, 4, EQUERRY, EAST)="
\\

A Scrabble Board has 15 rows and 15 columns, so ROW = 0,1,...,14, and COL = 0,1,...,14. The ROW and COL indicate where the word formed begins, NOT necessarily where you placed your first tile. For example, if ``new" is on the board and you play ``renew" then ROW and COL are where the `r' in ``renew" appears. 

In rare cases this can appear ambiguous, for example: 

\begin{verbatim}
SCRBL_RACK: PUMMEL*
SCRBL_BOARD:
_ _ _ _ _ _ _ _ _ _ _ _ _ _ _
_ _ _ _ _ _ _ _ _ _ _ _ _ _ _
_ _ _ _ _ _ _ _ _ _ _ _ _ _ _
_ _ _ _ _ _ _ _ _ _ _ _ _ _ _
_ _ _ _ _ _ _ _ _ _ _ _ _ _ _
_ _ _ _ _ _ _ _ _ _ _ _ _ _ _
_ _ _ _ _ _ _ _ _ _ _ _ _ _ _
_ _ _ _ _ _ _ H A M _ _ _ _ _
_ _ _ _ _ _ _ A _ A _ _ _ _ _
_ _ _ _ _ _ _ M A _ _ _ _ _ _
_ _ _ _ _ _ _ _ _ _ _ _ _ _ _
_ _ _ _ _ _ _ _ _ _ _ _ _ _ _
_ _ _ _ _ _ _ _ _ _ _ _ _ _ _
_ _ _ _ _ _ _ _ _ _ _ _ _ _ _
_ _ _ _ _ _ _ _ _ _ _ _ _ _ _
\end{verbatim}

You can play a `P' to form MAP twice. Which `M' is the start of your play? Don't worry about it, either one will work. 

\section{Dictionary}
You will use the Official Scrabble Player's 2nd Edition Dictionary. It is available as a plaintext file at \verb=somedir/dict/OSPDv2.txt=

\section{Letter and Word Bonuses}
Your AI must, of course, calculate points with the bonuses taken into account. The layout of a Scrabble board can be easily googled, or you can just parse the following text: 


\end{document} 
